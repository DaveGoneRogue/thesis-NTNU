% Better comment extension for Vscode colors these comments differently
% Normal comment color
% * Important information
% ! ALERT
% ? Question
% TODO stuff to do
% // This is strikethrough


MY NOTEPAD, this is not imported anywhere

%! Vet ikke hvor jeg vil ha denne. 
\begin{remark}
    We can also say that the pair $\brac{I^d,\Lambda}$ is orthogonal. 
\end{remark} 


%* Google Scholar search of S. Pedersen
https://scholar.google.com/citations?view_op=list_works&hl=en&hl=en&user=f9odLOsAAAAJ
%* ———————


%* Old tiling definition in two variations
\begin{definition}[Tiling set]
    Let $\Omega \subset \R^d$ be a subset with nonzero measure, and consider a set $\Lambda \subset \R^d$. If $T(\Lambda)$ covers $\R^d$ up to measure zero, and if all intersections 
    \begin{equation*}  %* NON OVERLAPPING
        (\Omega+\lambda) \cap (\Omega+\lambda')
    \end{equation*}
    for $\lambda\neq \lambda'$ in $\Lambda$ have measure zero, then $\Omega$ is called a \emph{tile}, and $\Lambda$ is called a \emph{tiling set} for $\Omega$. We say that $(\Omega, \Lambda)$ is a \emph{tiling pair}. 
\end{definition}
\begin{definition}  %* Sigrid would not want to use the above rather than this one
    Let $\Omega \subset \R^d$ be a subset with nonzero measure, and consider a set $\Lambda \subseteq \R^d$. If the following two conditions are satisfied, then $\Omega$ is called a \emph{tile}, and $\Lambda$ is called a \emph{tiling set} for $\Omega$. We say that $(\Omega, \Lambda)$ is a \emph{tiling pair}. 
    \begin{itemize}
        \item If $T(\Lambda)$ cover $\R^d$ up to measure zero. That is,  %* Cover the whole space
        \begin{equation*}
            \bigcup_{\lambda \in \Lambda} (\Omega + \lambda) = \R^d
        \end{equation*}
        \item If all intersections of $(\Omega+\lambda) \cap (\Omega+\lambda')$ for $\lambda\neq \lambda'$ in $\Lambda$ have measure zero. %* Mutually NON-OVERLAPPING 
    \end{itemize}
\end{definition}
%* ———————
