\documentclass[../thesis.tex]{subfiles}

\begin{document}
We follow Lurie \cite{HTT} closely.
We write $\Open(X)$ for the partial order of open subsets of a topological space $X$.
Let's define sheaves and K-sheaves.
\begin{definition}[{\cite[Definition 7.3.3.1]{HTT}}]\label{sheaf_on_top}
    Let $X \in \cat{Top}$ and $\C$ an $\infty$-category.
    We define a $\C$-valued sheaf on $X$ to be a presheaf $\F: \Open(X)^{op} \to \C$ such that, for every $U\subseteq X$ and every covering sieve $\Open(X)_{/U}^{(0)} \subseteq \Open(X)_{/U}$,
    $$\Nerve(\Open(X)_{/U}^{(0)})^{\triangleright} \hookrightarrow \Nerve(\Open(X)_{/U})^{\triangleright} \rightarrow N(\Open(X))\xrightarrow{\F}\C^{op}$$
    is a colimit diagram.
\end{definition}
\begin{remark}
    This is almost exactly definition of left Kan extension! Is it true that a presheaf is a sheaf if (and only if??) it is a left Kan extension of every possible covering sieve on $X$?
\end{remark}
We will write $\Presh(X, \C)$ for the $\infty$-category $\mathrm{Fun}(\Open(X)^{op}, \C)$ of $\C$-valued presheaves on $X$.
Moreover, we will write $\Shv(X;\C)$ for the full subcategory of $\Presh(X;\C)$ spanned by the $\C$-valued sheaves on $X$.
Whenever we write $\Shv(X)$ without specifying the target category $\C$, we will always mean sheaves valued in spaces, i.e. $\Shv(X;\Spaces)$.
\subsection{Sheaves on locally compact spaces}
In this section we will show that for locally compact Hausdorff spaces there is an equivalence of $\infty$-categories between $\Shv(X;\C)$ and $\Shv_{\K}(X;\C)$ where the latter denote so-called $\K$-sheaves and $\C$ is a presentable $\infty$-category with left exact filtered colimits.
These are sheaves defined on the collection of compact subsets instead of the opens.
Classically it is known that sheaves of sets on such spaces are determined by compact subsets as well as the opens.
\todo{Expand on this with references.}
\begin{definition}
    For a locally compact Hausdorff space $X$, we write $\K(X)$ for its collection of compact subsets.
\end{definition}
\begin{definition}
    If $K, K' \subseteq X$, we write $K \Subset K'$ if there exists an open subset $U\subseteq X$ between $K$ and $K'$, i.e. $K \subseteq U \subseteq K'$.
\end{definition}
\begin{definition}
    If $K\subseteq X$ is compact, we write $\K_{K\Subset}(X)$ for the set $\{ K'\in \K(X) | K \Subset K' \}$.
\end{definition}
\begin{definition}
    A presheaf $\F:\Nerve(\K(X))^{op}\to \C$ is a $\K$-sheaf if it satisfies the following:
    \begin{enumerate}
        \item $\F(\emptyset)$ is final
        \item For every pair $K,K'\in \K(X)$, the diagram
              \[\begin{tikzcd}
                      {\F(K \cup K')} && {\F(K)} \\
                      \\
                      {\F(K')} && {\F(K\cap K')}
                      \arrow[from=1-3, to=3-3]
                      \arrow[from=3-1, to=3-3]
                      \arrow[from=1-1, to=3-1]
                      \arrow["\lrcorner"{anchor=center, pos=0.125}, draw=none, from=1-1, to=3-3]
                      \arrow[from=1-1, to=1-3]
                  \end{tikzcd}\]
              is a pullback in $\C$.
        \item For each $K\in \K(X)$, $\F(K)$ is a colimit of $\F|\Nerve(\K_{K\Subset}(X))^{op}$.
    \end{enumerate}
\end{definition}
\begin{definition}
    We denote the full subcategory of $\Presh(\Nerve(\K(X)); \C)$ by $\Shv_{\K}(X;\C)$.
\end{definition}
\begin{lemma}[{\cite[][Lemma 7.3.4.8]{HTT}}]
    Let $X$ be locally compact and Hausdorff and $\C$ be a presentable $\infty$-category with left exact filtered colimits.
    Let $\W$ be an open cover of $X$ and denote by $\K_{\W}(X)$ the compact subsets of $X$ that are contained in some element of $\W$.
    Any $\K$-sheaf $\F\in \Shv_{\K}(X;\C)$ is a right Kan extension of $\F|\Nerve(\K_{\W}(X))^{op}$.
\end{lemma}
\begin{theorem}[{\cite[][Theorem 7.3.4.9]{HTT}}]
    Let $X$ be locally compact and Hausdorff and $\C$ a presentable $\infty$-category with left exact filtered colimits.
    Let $\F : \Nerve(\Open(X) \cup \K(X))^{op} \to \C$. TFAE:
    \begin{enumerate}[]
        \item The presheaf $\F_\K:= \F|\Nerve(\K(X))^{op}$ is a $\K$-sheaf, and $\F$ is a right Kan extension of $\F_\K$.
        \item The presheaf $\F_{\Open}:=\F|\Nerve(\Open(X))^{op}$ is a sheaf, and $\F$ is a left Kan extension of $\F_{\Open}$.
    \end{enumerate}
\end{theorem}
\begin{remark}
    I'm pretty sure we must regard the union $\Open(X) \cup \K(X)$ as a poset contained in the powerset of $X$.
\end{remark}
\begin{proof}
    We start by assuming the first condition and want to show that $F$ is left Kan extended from $\Nerve({\Open}(X))^{op}$.
    By definition we want to show that
    \[
        (\Nerve(\Open(X)^{op})_{/K})^{\triangleright} \hookrightarrow (\Nerve(\Open(X)\cup \K(X))_{/K}^{op})^{\triangleright} \xrightarrow{c}\Nerve(\Open(X) \cup \K(X))^{op}\xrightarrow{\F}\C
    \]
    is a colimit diagram in $\C$.
    The assumption that $\F_{\K}$ is a $\K$-sheaf means that For each $K\in \K(X)$, $\F_{\K}(K)$ is a colimit of $\F|\Nerve(\K_{K\Subset}(X))^{op}$.
    We will "transfer" this colimit to the colimit we want by cofinal maps
    \[
        (\Nerve(\Open(X)^{op})_{/K}) \xrightarrow{p} (\Nerve(\Open(X)\cup \K(X))_{/K}^{op}) \xleftarrow{p'}\Nerve(\K_{K\Subset}(X))^{op}.
    \]
    Recall that by \ref{superlemma} checking cofinality reduces to checking weak contractibility of certain simplicial sets.
    For $p$ we must check $(\Nerve(\Open(X)^{op})_{/K}) \times_{(\Nerve(\Open(X)\cup \K(X))_{/K}^{op})} (\Nerve(\Open(X)\cup \K(X))_{/K}^{op})_{K'/}$ is weakly contractible for every $K' \in (\Nerve(\Open(X)\cup \K(X))_{/K}^{op})$.
    This is the simplicial set obtained by taking the nerve of the partially ordered set$\{U \in \Open(X) | K\subseteq U \subseteq K'\}$.
    By \cite[Lemma 5.3.1.20]{HTT} filtered $\infty$-categories are weakly contractible, and our partially ordered set is filtered as it is nonempty and stable under finite union and taking nerves preserve the property of being filtered.
    The simplicial set $\Nerve(\{K'' | K \subseteq K'' \subseteq K'\})$ is weakly contractible by the exact same argument, and hence $p$ and $p'$ are cofinal maps.
    By cofinality of $p$ and $p'$, the diagram
    \[
        (\Nerve(\Open(X)^{op})_{/K})^{\triangleright} \hookrightarrow (\Nerve(\Open(X)\cup \K(X))_{/K}^{op})^{\triangleright} \xrightarrow{c}\Nerve(\Open(X) \cup \K(X))^{op}\xrightarrow{\F}\C
    \]
    is a colimit diagram if and only if
    \[
        (\Nerve(\K(X)^{op}_{\Subset K}))^{\triangleright} \hookrightarrow (\Nerve(\Open(X)\cup \K(X))_{/K}^{op})^{\triangleright} \xrightarrow{c}\Nerve(\Open(X) \cup \K(X))^{op}\xrightarrow{\F}\C
    \]
    is a colimit diagram, which it is by the assumption that $F_{\K}$ is a $\K$-sheaf.
    % \[\begin{tikzcd}
    %         {\Nerve(\Open(X)_{K\subseteq})^{op}} && {\Nerve(\Open(X)_{K\subseteq}\cup \K(X)_{K\Subset})^{op}} && {\Nerve(\K(X)_{K\Subset})^{op}} \\
    %         \\
    %         {\Nerve(\Open(X)_{K\subseteq}^{op})^{\triangleright}} && {\Nerve(\Open_{K\subseteq}(X)^{op}\cup \K(X)_{K\Subset}^{op})^{\triangleright}} && {\Nerve(\K(X)_{K\Subset}^{op})^{\triangleright}} \\
    %         \\
    %         && {\Nerve(\Open(X)\cup \K(X))^{op}} \\
    %         \\
    %         && \C
    %         \arrow["p", from=1-1, to=1-3]
    %         \arrow["{p'}"', from=1-5, to=1-3]
    %         \arrow[from=1-1, to=3-1]
    %         \arrow[from=3-1, to=3-3]
    %         \arrow[from=3-5, to=3-3]
    %         \arrow[from=1-3, to=3-3]
    %         \arrow[from=3-3, to=5-3]
    %         \arrow["\F", from=5-3, to=7-3]
    %         \arrow["{\psi'}"', from=3-5, to=7-3]
    %         \arrow["\psi", from=3-1, to=7-3]
    %         \arrow[from=1-5, to=3-5]
    %     \end{tikzcd}\]


\end{proof}
\end{document}